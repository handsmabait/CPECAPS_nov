\section{Sensor Choice, Representation, and Robustness}
A study by Tölgyessy, Dekan, and Chovanec (2021) demonstrated that Kinect-family depth sensors produce explicit 3D skeletons and give higher joint fidelity in controlled settings, but the accuracy falls with occlusion, off-axis views, and increased distance. Zhang et al. (2020) described MediaPipe, which yields compact 2D/3D landmark coordinates from ordinary RGB cameras and runs in real time on mobile devices. Therefore, designers often choose landmarks for rapid, lightweight prototypes and mobile deployment, and reserve depth or IR systems for installation-grade fidelity when hardware is available. To reduce real-world failure modes, practitioners apply photometric and background augmentation and synthetic occlusions during training, and they add a short calibration step so system metrics align with an individual user’s range of motion.

\section{Temporal Alignment and Scoring}
Dance is a temporal activity and should be compared as a sequence rather than as isolated frames. Yu and Xiong (2019) demonstrate that Dynamic Time Warping (DTW) can align noisy, tempo-varying Kinect skeleton sequences and convert DTW distances into meaningful performance scores. Rallis et al. (2019) apply DTW to choreographic trajectories and show it can match patterns across high-precision (VICON) and low-cost (Kinect) capture systems. Thus, a practical scoring pipeline first aligns sequences with DTW (or a constrained variant) and then evaluates local spatial metrics such as joint-angle differences or normalized trajectory distances to produce interpretable, per-segment correctness scores.

\section{Real-Time Feedback, Segmentation, and Pedagogy}
Lin (2015) finds that immediate, clear feedback in dance exergames improves engagement and supports learning. Zhang et al. (2020) show that on-device landmark extraction can run at real-time rates suitable for low-latency feedback. Combining these results suggests a two-tier runtime design: use a fast, coarse matcher (enabled by on-device landmarks) for instant cues, and run a slower, higher-precision alignment and scoring pass for final grading. Breaking choreography into short labeled segments also simplifies alignment and reduces error accumulation; Rallis et al. (2019) illustrate that segment- or trajectory-level matching better supports choreographic retrieval and per-segment feedback.

\section{Accessibility, Personalization, and Evaluation}
Yu and Xiong (2019) convert DTW distances into calibrated percentage scores, which supports per-user calibration and comparison against an individualized baseline. Tölgyessy et al. (2021) recommend measuring sensor-level metrics such as joint error and dropout rates when choosing a capture modality. Therefore, system designs should include adjustable sensitivity, alternate gesture mappings, and user profiles, and evaluation should combine sensor metrics (joint error, dropout, latency) with human-centered measures (perceived accuracy, engagement, and learning gain) to justify architecture and scoring choices.



% ---- Technical standards (ME) ----
\renewcommand{\arraystretch}{1.05}
\setlength{\tabcolsep}{4pt}

\begin{SingleSpace}
\begin{longtable}{|p{0.28\textwidth}|p{0.34\textwidth}|p{0.34\textwidth}|}
\caption{Technical standards (ME) -- scope and compliance justification}
\label{tab:technical_standards_missing} \\

\hline
\textbf{Standard / Regulation} & \textbf{Scope of Use in the System} & \textbf{Compliance Justification} \\ \hline
\endfirsthead

\multicolumn{3}{c}{{\bfseries Table \thetable\ (continued)}} \\ \hline
\textbf{Standard / Regulation} & \textbf{Scope of Use in the System} & \textbf{Compliance Justification} \\ \hline
\endhead

\hline \multicolumn{3}{r}{{Continued on next page}} \\ \hline
\endfoot

\hline
\endlastfoot

\emph{ISO 9241-210: Human-centered system design} &
UI design and user interaction &
Ensures user comfort and reduces fatigue during dance learning. \\ \hline

\emph{IEEE 802.11: Wi-Fi communication} &
If remote database or cloud storage is used &
Ensures interoperability and stable streaming between client and remote endpoints. \\ \hline

\emph{ISO 27001: Data privacy \& security} &
Storage and handling of video recordings &
Prevents unauthorized access to personal video data and enforces secure storage practices. \\ \hline

\emph{ISO 25010: Software quality characteristics} &
Reliability, maintainability, usability &
Used as a quality benchmark during evaluation and acceptance testing. \\ \hline

\emph{IEEE 754: Floating-point calculations} &
Pose and angle computations &
Ensures mathematical consistency and predictable numerical behaviour across platforms. \\ \hline

\end{longtable}
\end{SingleSpace}
% ---- end technical standards table ----




% ---- C4: Environmental & Safety standards ----
\renewcommand{\arraystretch}{1.05}
\setlength{\tabcolsep}{4pt}

\begin{SingleSpace}
\begin{longtable}{|p{0.28\textwidth}|p{0.66\textwidth}|}
\caption{Environmental \& Safety standards and their application in the project}
\label{tab:env_safety_compact} \\

\hline
\textbf{Standard / Regulation} & \textbf{Application} \\ \hline
\endfirsthead

\multicolumn{2}{c}{{\bfseries Table \thetable\ (continued)}} \\ \hline
\textbf{Standard / Regulation} & \textbf{Application} \\ \hline
\endhead

\hline \multicolumn{2}{r}{{Continued on next page}} \\ \hline
\endfoot

\hline
\endlastfoot

\emph{RA 9003: Ecological Solid Waste Management Act} &
Limits hardware waste; project reuses existing webcams and peripherals where possible to reduce e-waste and disposal burden. \\ \hline

\emph{ISO 14001: Environmental Management System} &
Guides procurement and lifecycle decisions to ensure minimal environmental impact when selecting cameras, computers, and consumables. \\ \hline

\emph{ISO 45001: Occupational health \& safety} &
Protects users and participants performing physical activity by mandating risk assessment, safe spaces (non-slip flooring), and emergency procedures. \\ \hline

\emph{IEC 60950-1: IT equipment electrical safety} &
Ensures safe usage of laptops, webcams, power supplies, and peripherals during prolonged sessions to prevent electrical hazards. \\ \hline

\end{longtable}
\end{SingleSpace}
% ---- end environmental & safety table ----



