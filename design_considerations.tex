\section{Sensor Choice, Representation, and Robustness}
A study by Tölgyessy, Dekan, and Chovanec (2021) demonstrated that Kinect-family depth sensors produce explicit 3D skeletons and give higher joint fidelity in controlled settings, but the accuracy falls with occlusion, off-axis views, and increased distance. Zhang et al. (2020) described MediaPipe, which yields compact 2D/3D landmark coordinates from ordinary RGB cameras and runs in real time on mobile devices. Therefore, designers often choose landmarks for rapid, lightweight prototypes and mobile deployment, and reserve depth or IR systems for installation-grade fidelity when hardware is available. To reduce real-world failure modes, practitioners apply photometric and background augmentation and synthetic occlusions during training, and they add a short calibration step so system metrics align with an individual user’s range of motion.

\section{Temporal Alignment and Scoring}
Dance is a temporal activity and should be compared as a sequence rather than as isolated frames. Yu and Xiong (2019) demonstrate that Dynamic Time Warping (DTW) can align noisy, tempo-varying Kinect skeleton sequences and convert DTW distances into meaningful performance scores. Rallis et al. (2019) apply DTW to choreographic trajectories and show it can match patterns across high-precision (VICON) and low-cost (Kinect) capture systems. Thus, a practical scoring pipeline first aligns sequences with DTW (or a constrained variant) and then evaluates local spatial metrics such as joint-angle differences or normalized trajectory distances to produce interpretable, per-segment correctness scores.

\section{Real-Time Feedback, Segmentation, and Pedagogy}
Lin (2015) finds that immediate, clear feedback in dance exergames improves engagement and supports learning. Zhang et al. (2020) show that on-device landmark extraction can run at real-time rates suitable for low-latency feedback. Combining these results suggests a two-tier runtime design: use a fast, coarse matcher (enabled by on-device landmarks) for instant cues, and run a slower, higher-precision alignment and scoring pass for final grading. Breaking choreography into short labeled segments also simplifies alignment and reduces error accumulation; Rallis et al. (2019) illustrate that segment- or trajectory-level matching better supports choreographic retrieval and per-segment feedback.

\section{Accessibility, Personalization, and Evaluation}
Yu and Xiong (2019) convert DTW distances into calibrated percentage scores, which supports per-user calibration and comparison against an individualized baseline. Tölgyessy et al. (2021) recommend measuring sensor-level metrics such as joint error and dropout rates when choosing a capture modality. Therefore, system designs should include adjustable sensitivity, alternate gesture mappings, and user profiles, and evaluation should combine sensor metrics (joint error, dropout, latency) with human-centered measures (perceived accuracy, engagement, and learning gain) to justify architecture and scoring choices.


\renewcommand{\arraystretch}{1.0}
\setlength{\tabcolsep}{6pt}

\begin{SingleSpace}
\begin{longtable}{|p{0.30\textwidth}|p{0.35\textwidth}|p{0.35\textwidth}|}
\caption{Technical standards and their scope / justification in the system} 
\label{tab:technical_standards} \\

\hline
\textbf{Standard / Regulation} & \textbf{Scope of Use in the System} & \textbf{Compliance Justification} \\ \hline
\endfirsthead

\multicolumn{3}{c}{{\bfseries Table \thetable\ (continued)}} \\ \hline
\textbf{Standard / Regulation} & \textbf{Scope of Use in the System} & \textbf{Compliance Justification} \\ \hline
\endhead

\hline \multicolumn{3}{r}{{Continued on next page}} \\ \hline
\endfoot

\hline
\endlastfoot

\emph{ISO 9241-210} (Human-centered system design) &
UI design and user interaction &
Ensures user comfort, reduces fatigue, and guides iterative usability-driven design for dance learning. \\ \hline

\emph{IEEE 802.11} (Wi-Fi communication) &
If remote database or cloud storage is used &
Ensures interoperability and stable streaming between client and cloud/storage endpoints. \\ \hline

\emph{ISO 27001 / GDPR} (Data privacy \& security) &
Storage and handling of video recordings and personal data &
Frameworks to prevent unauthorized access, mandate secure storage, and ensure legal compliance for personal video data. \\ \hline

\emph{ISO 25010} (Software quality characteristics) &
Reliability, maintainability, usability, performance, etc. &
Used as a quality benchmark for evaluation, test criteria, and acceptance checks during development. \\ \hline

\emph{IEEE 754} (Floating-point calculations) &
Pose and angle computations /
numerical processing &
Ensures mathematical consistency and predictable numerical behaviour across platforms. \\ \hline

\emph{ASHRAE 55} (Comfort during motion activity) — optional (ME) &
Physical room environment for motion studies &
Provides guidance on thermal and spatial comfort to ensure safe operating conditions during dance experiments. \\ \hline

\end{longtable}
\end{SingleSpace}


\renewcommand{\arraystretch}{1.0}
\setlength{\tabcolsep}{6pt}

\begin{SingleSpace}
\begin{longtable}{|p{0.30\textwidth}|p{0.35\textwidth}|p{0.35\textwidth}|}
\caption{Environmental and safety standards relevant to the system and experimental setup}
\label{tab:env_safety_standards} \\

\hline
\textbf{Standard / Regulation} & \textbf{Scope of Use in the System} & \textbf{Compliance Justification} \\ \hline
\endfirsthead

\multicolumn{3}{c}{{\bfseries Table \thetable\ (continued)}} \\ \hline
\textbf{Standard / Regulation} & \textbf{Scope of Use in the System} & \textbf{Compliance Justification} \\ \hline
\endhead

\hline \multicolumn{3}{r}{{Continued on next page}} \\ \hline
\endfoot

\hline
\endlastfoot

\emph{ISO 14001} (Environmental management) &
Device lifecycle, laboratory waste handling, energy consumption of workstations &
Ensures systematic control of environmental impacts (e.g., energy use, e-waste) and guides sustainable disposal of cameras, GPUs, and electronics. \\ \hline

\emph{WEEE Directive / local e-waste regulation} &
End-of-life handling and recycling of electronic components &
Mandates proper collection and recycling routes for electrical equipment to reduce landfill and hazardous waste. \\ \hline

\emph{RoHS (Restriction of Hazardous Substances)} &
Selection and procurement of electronic hardware (sensors, cables, PCBs) &
Limits hazardous materials (lead, mercury, cadmium) in devices used, reducing toxic exposure and disposal risks. \\ \hline

\emph{ISO 45001} (Occupational health \& safety management) &
Laboratory and user testing sessions; staff and participant safety procedures &
Provides a framework to identify hazards, implement controls (e.g., slip/trip/fall mitigation for dance experiments), and protect participants and researchers. \\ \hline

\emph{IEC 62368-1} (Safety requirements for audio/video and ICT equipment) &
Electrical/electronic safety of cameras, computers, chargers, and peripherals used in the setup &
Specifies safe design and installation practices to prevent electrical shock, overheating, and fire hazards in experimental rigs. \\ \hline

\emph{ISO 12100} (Safety of machinery -- risk assessment) &
Risk assessment for experimental apparatus and movement spaces (e.g., bamboo poles, floor surface) &
Guides formal hazard analysis and risk reduction for moving elements, trip hazards, and equipment placement during Tinikling trials. \\ \hline

\emph{Local Fire Code / NFPA 101 (Life Safety Code)} &
Venue occupancy, egress, electrical wiring, and emergency procedures during data-collection sessions &
Ensures safe occupant loads, clear egress routes, and appropriate electrical/fire safety provisions for on-site testing. \\ \hline

\emph{IEC 60529 (IP code)} &
Selection/deployment of hardware in semi-outdoor or dusty environments &
Specifies ingress protection levels to select housings and cameras appropriate for environmental exposure, reducing device failure and contamination risks. \\ \hline

\emph{Exercise safety guidance (e.g., ACSM / WHO physical activity recommendations)} &
Participant screening, warm-up/cooldown protocols, and safe activity limits during user studies &
Provides evidence-based guidance to minimise injury risk during dance trials and to design ethical, safe participant protocols. \\ \hline

\end{longtable}
\end{SingleSpace}

