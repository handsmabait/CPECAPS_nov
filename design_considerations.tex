\section{Sensor Choice, Representation, and Robustness}
A study by Tölgyessy, Dekan, and Chovanec (2021) demonstrated that Kinect-family depth sensors produce explicit 3D skeletons and give higher joint fidelity in controlled settings, but the accuracy falls with occlusion, off-axis views, and increased distance. Zhang et al. (2020) described MediaPipe, which yields compact 2D/3D landmark coordinates from ordinary RGB cameras and runs in real time on mobile devices. Therefore, designers often choose landmarks for rapid, lightweight prototypes and mobile deployment, and reserve depth or IR systems for installation-grade fidelity when hardware is available. To reduce real-world failure modes, practitioners apply photometric and background augmentation and synthetic occlusions during training, and they add a short calibration step so system metrics align with an individual user’s range of motion.

\section{Temporal Alignment and Scoring}
Dance is a temporal activity and should be compared as a sequence rather than as isolated frames. Yu and Xiong (2019) demonstrate that Dynamic Time Warping (DTW) can align noisy, tempo-varying Kinect skeleton sequences and convert DTW distances into meaningful performance scores. Rallis et al. (2019) apply DTW to choreographic trajectories and show it can match patterns across high-precision (VICON) and low-cost (Kinect) capture systems. Thus, a practical scoring pipeline first aligns sequences with DTW (or a constrained variant) and then evaluates local spatial metrics such as joint-angle differences or normalized trajectory distances to produce interpretable, per-segment correctness scores.

\section{Real-Time Feedback, Segmentation, and Pedagogy}
Lin (2015) finds that immediate, clear feedback in dance exergames improves engagement and supports learning. Zhang et al. (2020) show that on-device landmark extraction can run at real-time rates suitable for low-latency feedback. Combining these results suggests a two-tier runtime design: use a fast, coarse matcher (enabled by on-device landmarks) for instant cues, and run a slower, higher-precision alignment and scoring pass for final grading. Breaking choreography into short labeled segments also simplifies alignment and reduces error accumulation; Rallis et al. (2019) illustrate that segment- or trajectory-level matching better supports choreographic retrieval and per-segment feedback.

\section{Accessibility, Personalization, and Evaluation}
Yu and Xiong (2019) convert DTW distances into calibrated percentage scores, which supports per-user calibration and comparison against an individualized baseline. Tölgyessy et al. (2021) recommend measuring sensor-level metrics such as joint error and dropout rates when choosing a capture modality. Therefore, system designs should include adjustable sensitivity, alternate gesture mappings, and user profiles, and evaluation should combine sensor metrics (joint error, dropout, latency) with human-centered measures (perceived accuracy, engagement, and learning gain) to justify architecture and scoring choices.
