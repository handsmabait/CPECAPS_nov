% methodology.tex
% Included by document.tex which already contains \chapter{Methodology}

\section{Methodology}

\begin{figure}[htbp]
    \centering
    \includegraphics[width=0.8\textwidth]{methodology.png}
    \caption{Methodology Block Diagram}
    \label{fig:methodology_flowchart}
\end{figure}

\subsection{Methodology Overview}
This project develops a desktop real-time pose-estimation application for Tinikling learning. The pipeline comprises (1) dataset collection and annotation, (2) real-time landmark detection using MediaPipe with OpenCV preprocessing, (3) model robustness improvements via augmentation and fine-tuning, (4) a per-segment scoring and feedback engine, and (5) system evaluation and user studies for performance and usability.

\begin{figure}[htbp]
    \centering
    \includegraphics[width=\linewidth]{methoddia.png}
    \caption{System Diagram of the Real-time Tinikling Learning Application}
    \label{fig:system}
\end{figure}

\begin{figure}[htbp]
    \centering
    \includegraphics[width=\linewidth]{final_method_diagram.png}
    \caption{System Diagram of the Real-time Tinikling Learning Application}
    \label{fig:system}
\end{figure}

% --- Methodology table: single-spaced, multi-page longtable with objectives/methods/locations ---
\renewcommand{\arraystretch}{1.0} % tighten vertical spacing
\setlength{\tabcolsep}{6pt}      % tighten horizontal padding

\begin{SingleSpace}
\begin{longtable}{|p{0.20\textwidth}|p{0.64\textwidth}|p{0.12\textwidth}|}
\caption{Summary of methods for reaching the objectives}
\label{tab:methods_per_objective} \\

\hline
\textbf{Objectives} & \textbf{Methods} & \textbf{Locations} \\ \hline
\endfirsthead

\multicolumn{3}{c}{{\bfseries Table \thetable\ (continued)}} \\ \hline
\textbf{Objectives} & \textbf{Methods} & \textbf{Locations} \\ \hline
\endhead

\hline \multicolumn{3}{r}{{Continued on next page}} \\ \hline
\endfoot

\hline
\endlastfoot

\hline
\textbf{GO:} To develop a real-time pose estimation-based Tinikling learning application. &
\begin{enumerate}
  \item Develop a desktop application integrating pose estimation, scoring, and feedback modules.
  \item Utilize MediaPipe + OpenCV for pose detection, integrated with a GUI framework.
  \item Document architecture, usage, and installation following software engineering practices.
\end{enumerate}
& N/A \\ \hline

\textbf{SO1:} To develop a real-time pose estimation pipeline that captures dancers' movements using a webcam, detects key skeletal landmarks, and analyzes Tinikling steps with $\geq 30$ fps processing speed and $\geq 70\%$ detection accuracy. &
\begin{enumerate}
  \item Use MediaPipe Pose for skeletal landmark detection in real time.
  \item Optimize frame processing via OpenCV preprocessing and efficient landmark extraction.
  \item Evaluate detection accuracy using collected test sequences and performance metrics.
\end{enumerate}
& $\geq 90\%$ detection accuracy; 30 fps \\ \hline

\textbf{SO2:} To make the pose estimation model robust to lighting, background clutter, and user variation through dataset collection and augmentation, while maintaining minimum pose detection accuracy of $85\%$. &
\begin{enumerate}
  \item Collect / create Tinikling dance videos under diverse lighting, backgrounds, and performer variations.
  \item Apply data augmentation (photometric, geometric, synthetic occlusions).
  \item Retrain / fine-tune the model and evaluate on a validation set to quantify improvements.
\end{enumerate}
& $\geq 85\%$ detection accuracy \\ \hline

\textbf{SO3:} To design and integrate a scoring and feedback system that aligns poses with reference choreographies, provides numerical scores (0--100) and step-by-step accuracy breakdown within $\leq 1$\,s after performance. &
\begin{enumerate}
  \item Implement per-segment accuracy scoring (DTW or constrained alignment + local spatial metrics).
  \item Build a choreography reference library with segmented Tinikling steps for alignment.
  \item Integrate UI feedback: immediate cues and post-performance breakdown.
\end{enumerate}
& Score range 0--100; feedback latency $\leq 1$\,s \\ \hline

\textbf{SO4:} To evaluate the system's performance and usability through controlled testing with at least 3 participants, measuring pose estimation accuracy, latency, and user satisfaction ($\geq 80\%$ positive feedback). &
\begin{enumerate}
  \item Conduct user testing sessions with participants performing choreographed sequences.
  \item Measure pose estimation accuracy, system latency, and feedback timing.
  \item Compile results into an evaluation report with recommendations for refinement.
\end{enumerate}
& $n \geq 10$ participants; $\geq 80\%$ positive feedback \\ \hline

\end{longtable}
\end{SingleSpace}
% --- end methodology table ---


\subsection{Dataset Collection and Annotation}
We collect Tinikling performances using consumer webcams across varied environments (lighting, backgrounds, participant clothing). Each recording is annotated with segment boundaries and ground-truth reference trajectories for the core Tinikling steps. Annotation files follow a simple CSV schema: frame index, timestamp, keypoint coordinates (x,y[,z if available]), and segment label.

\subsubsection{Real-time Pipeline (Implementation)}
\label{sec:implement}
The real-time pipeline components:
\begin{enumerate}
    \item \textbf{Capture \& Preprocessing:} Acquire frames from webcam at target frame rates; apply resizing, color normalization, and optional background subtraction using OpenCV.
    \item \textbf{Landmark Detection:} Run MediaPipe Pose to extract 2D/3D keypoints; post-process landmarks (smoothing, confidence thresholding).
    \item \textbf{Segmentation \& Alignment:} Detect segment boundaries (simple heuristics or learned segment classifier), then align performed segment to reference via DTW or constrained alignment.
    \item \textbf{Scoring \& Feedback:} Compute per-joint and per-segment metrics; convert distances to 0--100 scores, present instant cues (visual/audio) and detailed breakdowns in UI.
    \item \textbf{Logging \& Persistence:} Save session logs, computed metrics, and anonymized recordings for later analysis.
\end{enumerate}

\subsection{Model Robustness and Training}
To improve robustness:
\begin{itemize}
    \item Augment datasets with photometric (brightness/contrast), geometric (rotation, scale), and synthetic occlusion transforms.
    \item Perform k-fold validation and ablation studies to measure the effect of augmentations.
    \item Where appropriate, fine-tune a lightweight backbone (e.g., MediaPipe-compatible network) or add a small temporal refinement network (multi-frame fusion) for increased temporal stability.
\end{itemize}

\subsection{Scoring, Calibration, and UX}
Scoring converts aligned distances into interpretable percentages per segment:
\[
\text{score} = 100 \times \max\!\left(0, 1 - \frac{\text{normalized\_error}}{\text{threshold}}\right)
\]
Calibration includes per-user baseline capture (neutral stance and sample steps) to normalize per-joint tolerances. UI design emphasizes low-latency cues for learning (immediate feedback) and a post-run breakdown for correction.

\subsection{Evaluation Plan}
\begin{enumerate}
    \item \textbf{Automated metrics:} Detection accuracy (\%), MPJPE where available, processing fps, latency (ms).
    \item \textbf{User study:} $n\geq 10$ participants performing a standardized Tinikling routine; questionnaires to measure perceived accuracy, ease-of-use, and satisfaction. Target: $\geq 80\%$ positive feedback.
    \item \textbf{Robustness tests:} Evaluate under varied lighting, occlusion, and viewpoint conditions; measure drop in accuracy and suggest mitigations.
    \item \textbf{Report:} Compile results, run statistical tests where applicable, and provide actionable recommendations.
\end{enumerate}

\subsection{Deliverables}
\begin{itemize}
    \item Desktop application with installer and README (architecture, usage, install).
    \item Annotated dataset subset and reference choreography library.
    \item Evaluation report including metrics, user-study results, and recommendations.
    \item Source code release and simple reproducibility instructions.
\end{itemize}

\section{Summary}
This methodology outlines a practical pipeline to build and evaluate a real-time Tinikling learning tool: dataset creation, MediaPipe-based real-time detection with OpenCV optimizations, augmentation and fine-tuning for robustness, DTW-based alignment and scoring, and human-subject evaluation for usability and performance validation.

% end of methodology.tex
